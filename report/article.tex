\author{Andrea Jemmett, Chiel Kooijman}
\title{Evolving Behaviour for Thymio Robots using NEAT}
% XXX: Include interest-invoking words as real-time, online, etc
\documentclass{article}
\usepackage{graphicx,amsmath}

\begin{document}
	\maketitle
	\begin{abstract}
		We made robots do cool things.
		% FIXME: Make the abstract less abstract.
	\end{abstract}

	\section{Introduction} % {{{
	\label{sec:Introduction}
	In the field of Evolutionary Robotics (ER), several papers have been
	published over the last decades, but most were either concerned
	with evolving a robot controller in a virtual environment
	\cite{silva2015case, simoes1999evolutionary, sims1994evolving}, or more
	recently about evolving neural networks for behaviour on real robots, but
	only succesfully for limited tasks \cite{heinermanevolution,
	watson1999embodied}. In this paper we explore how more complex behaviour
	can be evolved on real robots using (Hyper?)NEAT \cite{stanley2002evolving}
	and using the right processing of input data. As far as the authors are
	aware this is the first time NEAT has been used to evolve behaviour on
	physical robots.
	% TODO: Something about energy and taking inspiration from life and RL?
	% TODO: and we show how we do impossibly amazing things and are much better
	% at everything than everyone else ever.
	In the next section we cover previous work on ER and NEAT. In section
	\ref{sec:method} we give a precise description on the hardware used, we
	explain the Obstacle Avoidance and Foraging tasks, including the reward
	model, the envorinment in which the robots were evaluated, and the
	algorithm and settings used for the experiments.
	In sections \ref{sec:setup} and \ref{sec:results} we cover the exact
	parameters used to run the experiments and their results respectively.
	Conclusions are drawn and further research suggested in the last section.
	% section Introduction }}}

	\section{Related Work} % {{{
	\label{sec:Related Work}
	\cite{heinermanevolution}
	\cite{silva2012odneat}
	\cite{stanley2002evolving, stanley2009hypercube, stanley2006real}
	% section Related Work }}}

	\section{System Description} % {{{
	\label{sec:method}
	\subsection{Robot} % {{{
	\label{sub:Robot}
	The Thymio II robot includes 7 Infra-Red (IR) proximity sensors able detect
	to obstacles of which are 5 in the front and two in the back (values
	between 0 and around 4500, where a higher value corresponds to a near
	obstacle).
	In the foraging task the robot is outfitted with a Raspberry Pi Camera
	Module, which streams 320x240 pixel images, and a piece of velcro with
	which it can grab the pucks. The camera is placed so that it can still see
	the puck when it has stuck to it while seeing as far forward as possible.
	The robot can move through two differential wheels, meaning that two
	different speeds (range between -300 and 300 for obstacle avoidance and
	-150 to 250 for foraging) can be set for each wheel. For the purpose of our
	research, we extend the standard setup with a more powerful logic board,
	wireless communication, and a high capacity battery. We use a Raspberry Pi
	B+ (credit card-sized single-board computer developed in the UK by the
	Raspberry Pi Foundation) that interacts with the Thymio sensors and
	actuators. A WiFi dongle (Edimax 150Mbps Wireless 802.11b/g/n nano USB WiFi
	Adapter model EW-7811Un) attached to the Raspberry Pi ensures communication
	between the robots. The power is given by a Verbatim Dual USB 12000 mAh
	battery that allows for a total experimental time of 10 hours.
	% FIXME maybe not copy this verbatim?
	% subsection Robot }}}

	\subsection{Environment} % {{{
	\label{sub:Environment}
	The robots operate in an arena of two meters by two and a half meters with
	inner and outer walls that act as obstacles to avoid.
	In the foraging task one of the corners of the arena is blue with white
	patches to counteract white-balance issues, and there are five green pucks
	with velcro sides placed in the arena.
	% subsection Environment }}}

	\subsection{Obstacle Avoidance} % {{{
	\label{sub:Obstacle Avoidance}
	blabla.
	% subsection Obstacle Avoidance }}}

	\subsection{Foraging} % {{{
	\label{sub:Foraging}
	In the foraging task the goal is to bring pucks to the goal area using the
	camera as often as possible within its evaluation time.

	\paragraph{Inputs} % {{{
	\label{par:Inputs}
	Inputs consist of the distance and angle to the closest pixel that shows
	the puck and the goal from the bottom centre of the image, and a binary
	flag representing whether the robot has the puck:
	(more explanation)

	recurrent
	% paragraph Inputs }}}

	\begin{itemize}
		\item inputs (dist, angle) for (puck, goal) and has\_puck
		\item energy
		\item fitness
		\item foraging parameters: MIN\_GOAL\_DIST, DECAYS, MAX\_STEPS
	\end{itemize}
	% xXX image: picture of robot + lego contraption
	% subsection Foraging }}}

	% section System Description }}}

	\section{Experimental Setup} % {{{
	\label{sec:setup}

	% section Experimental Setup }}}

	\section{Experimental Results} % {{{
	\label{sec:results}
	
	% section results }}}

	\section{Conclusion and Further Work} % {{{
	\label{sec:conclusion}
	Using energy helps limit evaluation time which makes evolving behaviour on
	physical robots more viable.
	% TODO: use experiments to show that this is actually true
	Calculating something akin to distance and angle is possible in real time.
	Limiting the number of input nodes makes learning faster. % TODO does it?

	In the future:\\
	\begin{itemize}
		\item Use Reinforcement Learning (DQN).\\
		\item Use more robots with a distributed system (e.g. odNEAT).\\
		\item Use Distributed Reinforcement Learning (i.e. sharing network
			weights).
	\end{itemize}
	% section conclusion }}}

\bibliographystyle{abbrv}
\bibliography{references}
\end{document}
